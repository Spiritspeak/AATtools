\documentclass[a4paper]{book}
\usepackage[times,inconsolata,hyper]{Rd}
\usepackage{makeidx}
\usepackage[utf8]{inputenc} % @SET ENCODING@
% \usepackage{graphicx} % @USE GRAPHICX@
\makeindex{}
\begin{document}
\chapter*{}
\begin{center}
{\textbf{\huge Package `AATtools'}}
\par\bigskip{\large \today}
\end{center}
\begin{description}
\raggedright{}
\inputencoding{utf8}
\item[Type]\AsIs{Package}
\item[Title]\AsIs{Tools for Analyzing the Approach-Avoidance Task}
\item[Version]\AsIs{0.0.1}
\item[Author]\AsIs{Sercan Kahveci}
\item[Description]\AsIs{Compute approach bias scores using different scoring algorithms,
compute bootstrapped and exact split-half reliability of the AAT,
and compute confidence intervals for individual AAT scores.}
\item[Depends]\AsIs{R (>= 3.6.1), magrittr, dplyr, doParallel}
\item[License]\AsIs{GPL-3}
\item[Encoding]\AsIs{UTF-8}
\item[BugReports]\AsIs{}\url{https://github.com/Spiritspeak/AATtools/issues}\AsIs{}
\item[LazyData]\AsIs{true}
\item[ByteCompile]\AsIs{true}
\item[RoxygenNote]\AsIs{7.1.0}
\end{description}
\Rdcontents{\R{} topics documented:}
\inputencoding{utf8}
\HeaderA{aat\_bootstrap}{Compute bootstrapped approach-bias scores}{aat.Rul.bootstrap}
\aliasA{plot.aat\_bootstrap}{aat\_bootstrap}{plot.aat.Rul.bootstrap}
\aliasA{print.aat\_bootstrap}{aat\_bootstrap}{print.aat.Rul.bootstrap}
%
\begin{Description}\relax
Compute bootstrapped approach-bias scores with confidence intervals.
\end{Description}
%
\begin{Usage}
\begin{verbatim}
aat_bootstrap(
  ds,
  subjvar,
  pullvar,
  targetvar = NULL,
  rtvar,
  iters,
  plot = T,
  include.raw = F,
  algorithm = c("aat_doublemeandiff", "aat_doublemediandiff", "aat_dscore",
    "aat_dscore_multiblock", "aat_regression", "aat_standardregression",
    "aat_doublemeanquotient", "aat_doublemedianquotient", "aat_singlemeandiff",
    "aat_singlemediandiff"),
  trialdropfunc = c("prune_nothing", "trial_prune_3SD", "trial_prune_SD_dropcases",
    "trial_recode_SD", "trial_prune_percent_subject", "trial_prune_percent_sample"),
  errortrialfunc = c("prune_nothing", "error_replace_blockmeanplus",
    "error_prune_dropcases"),
  ...
)

## S3 method for class 'aat_bootstrap'
print(x)

## S3 method for class 'aat_bootstrap'
plot(x)
\end{verbatim}
\end{Usage}
%
\begin{Arguments}
\begin{ldescription}
\item[\code{ds}] a longformat data.frame

\item[\code{subjvar}] Quoted name of the participant identifier column

\item[\code{pullvar}] Quoted name of the column indicating pull trials.
Pull trials should either be represented by 1, or by the second level of a factor.

\item[\code{targetvar}] Name of the column indicating trials featuring the target stimulus.
Target stimuli should either be represented by 1, or by the second level of a factor.

\item[\code{rtvar}] Name of the reaction time column.

\item[\code{iters}] Total number of desired iterations. At least 200 are required to get confidence intervals that make sense.

\item[\code{plot}] Plot the bias scores and their confidence intervals after computation is complete. This gives a good overview of the data.

\item[\code{algorithm}] Function (without brackets or quotes) to be used to compute AAT scores. See \LinkA{aat\_doublemeandiff}{aat.Rul.doublemeandiff} for a list of usable algorithms.

\item[\code{trialdropfunc}] Function (without brackets or quotes) to be used to exclude outlying trials in each half.
The way you handle outliers for the reliability computation should mimic the way you do it in your regular analyses.
It is recommended to exclude outlying trials when computing AAT scores using the mean double-dfference scores and regression scoring approaches,
but not when using d-scores or median double-difference scores.
\begin{itemize}

\item{} \code{prune\_nothing} excludes no trials (default)
\item{} \code{trial\_prune\_3SD} excludes trials deviating more than 3SD from the mean per participant.
\item{} \code{trial\_prune\_SD\_dropcases} removes trials deviating more than a specific number of standard deviations from the participant's mean,
and removes participants with an excessive percentage of outliers.
Required arguments:
\begin{itemize}

\item{} \code{trialsd} - trials deviating more than \code{trialsd} standard deviations from the participant's mean are excluded (optional; default is 3)
\item{} \code{maxoutliers} - participants with a higher percentage of outliers are removed from the data. (optional; default is .15)

\end{itemize}

\item{} \code{trial\_recode\_SD} recodes outlying reaction times to the nearest non-outlying value,
with outliers defined as reaction times deviating more than a certain number of standard deviations from the participant's mean. Required argument:
\begin{itemize}

\item{} \code{trialsd} - trials deviating more than this many standard deviations from the mean are classified as outliers.

\end{itemize}

\item{} \code{trial\_prune\_percent\_subject} and \code{trial\_prune\_percent\_sample} remove trials below and/or above certain percentiles,
on a subject-by-subject basis or sample-wide, respectively. The following arguments are available:
\begin{itemize}

\item{} \code{lowerpercent} and \code{uppperpercent} (optional; defaults are .01 and .99).

\end{itemize}


\end{itemize}


\item[\code{errortrialfunc}] Function (without brackets or quotes) to apply to an error trial.

\begin{itemize}

\item{} \code{prune\_nothing} removes no errors (default).
\item{} \code{error\_replace\_blockmeanplus} replaces error trial reaction times with the block mean, plus an arbitrary extra quantity.
If used, the following additional arguments are required:
\begin{itemize}

\item{} \code{blockvar} - Quoted name of the block variable (mandatory)
\item{} \code{errorvar} - Quoted name of the error variable, where errors are 1 or TRUE and correct trials are 0 or FALSE (mandatory)
\item{} \code{errorbonus} - Amount to add to the reaction time of error trials. Default is 0.6 (recommended by \code{Greenwald, Nosek, \& Banaji, 2003})

\end{itemize}

\item{} \code{error\_prune\_dropcases} removes errors and drops participants if they have more errors than a given percentage. The following arguments are available:
\begin{itemize}

\item{} \code{errorvar} - Quoted name of the error variable, where errors are 1 or TRUE and correct trials are 0 or FALSE (mandatory)
\item{} \code{maxerrors} - participants with a higher percentage of errors are excluded from the dataset. Default is .15.

\end{itemize}


\end{itemize}


\item[\code{...}] Other arguments, to be passed on to the algorithm or outlier rejection functions (see arguments above)
\end{ldescription}
\end{Arguments}
%
\begin{Value}
A list, containing bootstrapped bias scores, their variance, bootstrapped 95 percent confidence intervals,
the number of iterations, and a matrix of bias scores for each iteration.
\end{Value}
%
\begin{Author}\relax
Sercan Kahveci
\end{Author}
\inputencoding{utf8}
\HeaderA{aat\_compute}{Compute simple AAT scores}{aat.Rul.compute}
%
\begin{Description}\relax
Compute simple AAT scores, with optional outlier exclusion and error trial recoding.
\end{Description}
%
\begin{Usage}
\begin{verbatim}
aat_compute(
  ds,
  subjvar,
  pullvar,
  targetvar = NULL,
  rtvar,
  algorithm = c("aat_doublemeandiff", "aat_doublemediandiff", "aat_dscore",
    "aat_dscore_multiblock", "aat_regression", "aat_standardregression",
    "aat_doublemeanquotient", "aat_doublemedianquotient", "aat_singlemeandiff",
    "aat_singlemediandiff"),
  trialdropfunc = c("prune_nothing", "trial_prune_3SD", "trial_prune_SD_dropcases",
    "trial_recode_SD", "trial_prune_percent_subject", "trial_prune_percent_sample"),
  errortrialfunc = c("prune_nothing", "error_replace_blockmeanplus",
    "error_prune_dropcases"),
  ...
)
\end{verbatim}
\end{Usage}
%
\begin{Arguments}
\begin{ldescription}
\item[\code{ds}] a long-format data.frame

\item[\code{subjvar}] column name of subject variable

\item[\code{pullvar}] column name of pull/push indicator variable, must be numeric or logical (where pull is 1 or TRUE)

\item[\code{targetvar}] column name of target stimulus indicator, must be numeric or logical (where target is 1 or TRUE)

\item[\code{rtvar}] column name of reaction time variable

\item[\code{trialdropfunc}] Function (without brackets or quotes) to be used to exclude outlying trials in each half.
The way you handle outliers for the reliability computation should mimic the way you do it in your regular analyses.
It is recommended to exclude outlying trials when computing AAT scores using the mean double-dfference scores and regression scoring approaches,
but not when using d-scores or median double-difference scores.
\begin{itemize}

\item{} \code{prune\_nothing} excludes no trials (default)
\item{} \code{trial\_prune\_3SD} excludes trials deviating more than 3SD from the mean per participant.
\item{} \code{trial\_prune\_SD\_dropcases} removes trials deviating more than a specific number of standard deviations from the participant's mean,
and removes participants with an excessive percentage of outliers.
Required arguments:
\begin{itemize}

\item{} \code{trialsd} - trials deviating more than \code{trialsd} standard deviations from the participant's mean are excluded (optional; default is 3)
\item{} \code{maxoutliers} - participants with a higher percentage of outliers are removed from the data. (optional; default is .15)

\end{itemize}

\item{} \code{trial\_recode\_SD} recodes outlying reaction times to the nearest non-outlying value,
with outliers defined as reaction times deviating more than a certain number of standard deviations from the participant's mean. Required argument:
\begin{itemize}

\item{} \code{trialsd} - trials deviating more than this many standard deviations from the mean are classified as outliers.

\end{itemize}

\item{} \code{trial\_prune\_percent\_subject} and \code{trial\_prune\_percent\_sample} remove trials below and/or above certain percentiles,
on a subject-by-subject basis or sample-wide, respectively. The following arguments are available:
\begin{itemize}

\item{} \code{lowerpercent} and \code{uppperpercent} (optional; defaults are .01 and .99).

\end{itemize}


\end{itemize}


\item[\code{errortrialfunc}] Function (without brackets or quotes) to apply to an error trial.

\begin{itemize}

\item{} \code{prune\_nothing} removes no errors (default).
\item{} \code{error\_replace\_blockmeanplus} replaces error trial reaction times with the block mean, plus an arbitrary extra quantity.
If used, the following additional arguments are required:
\begin{itemize}

\item{} \code{blockvar} - Quoted name of the block variable (mandatory)
\item{} \code{errorvar} - Quoted name of the error variable, where errors are 1 or TRUE and correct trials are 0 or FALSE (mandatory)
\item{} \code{errorbonus} - Amount to add to the reaction time of error trials. Default is 0.6 (recommended by \code{Greenwald, Nosek, \& Banaji, 2003})

\end{itemize}

\item{} \code{error\_prune\_dropcases} removes errors and drops participants if they have more errors than a given percentage. The following arguments are available:
\begin{itemize}

\item{} \code{errorvar} - Quoted name of the error variable, where errors are 1 or TRUE and correct trials are 0 or FALSE (mandatory)
\item{} \code{maxerrors} - participants with a higher percentage of errors are excluded from the dataset. Default is .15.

\end{itemize}


\end{itemize}


\item[\code{...}] Other arguments, to be passed on to the algorithm or outlier rejection functions (see arguments above)
\end{ldescription}
\end{Arguments}
\inputencoding{utf8}
\HeaderA{aat\_splithalf}{Compute the bootstrapped split-half reliability for approach-avoidance task data}{aat.Rul.splithalf}
\aliasA{case\_prune\_3SD}{aat\_splithalf}{case.Rul.prune.Rul.3SD}
\aliasA{error\_prune\_dropcases}{aat\_splithalf}{error.Rul.prune.Rul.dropcases}
\aliasA{error\_replace\_blockmeanplus}{aat\_splithalf}{error.Rul.replace.Rul.blockmeanplus}
\aliasA{plot.aat\_splithalf}{aat\_splithalf}{plot.aat.Rul.splithalf}
\aliasA{print.aat\_splithalf}{aat\_splithalf}{print.aat.Rul.splithalf}
\aliasA{prune\_nothing}{aat\_splithalf}{prune.Rul.nothing}
\aliasA{trial\_prune\_3SD}{aat\_splithalf}{trial.Rul.prune.Rul.3SD}
\aliasA{trial\_prune\_percent\_sample}{aat\_splithalf}{trial.Rul.prune.Rul.percent.Rul.sample}
\aliasA{trial\_prune\_percent\_subject}{aat\_splithalf}{trial.Rul.prune.Rul.percent.Rul.subject}
\aliasA{trial\_prune\_SD\_dropcases}{aat\_splithalf}{trial.Rul.prune.Rul.SD.Rul.dropcases}
\aliasA{trial\_recode\_SD}{aat\_splithalf}{trial.Rul.recode.Rul.SD}
%
\begin{Description}\relax
Compute bootstrapped split-half reliability for approach-avoidance task data.
\end{Description}
%
\begin{Usage}
\begin{verbatim}
aat_splithalf(
  ds,
  subjvar,
  pullvar,
  targetvar = NULL,
  rtvar,
  iters,
  plot = T,
  include.raw = F,
  cluster = NULL,
  algorithm = c("aat_doublemeandiff", "aat_doublemediandiff", "aat_dscore",
    "aat_dscore_multiblock", "aat_regression", "aat_standardregression",
    "aat_doublemedianquotient", "aat_doublemeanquotient", "aat_singlemeandiff",
    "aat_singlemediandiff"),
  trialdropfunc = c("prune_nothing", "trial_prune_3SD", "trial_prune_SD_dropcases",
    "trial_recode_SD", "trial_prune_percent_subject", "trial_prune_percent_sample"),
  errortrialfunc = c("prune_nothing", "error_replace_blockmeanplus",
    "error_prune_dropcases"),
  casedropfunc = c("prune_nothing", "case_prune_3SD"),
  ...
)

## S3 method for class 'aat_splithalf'
print(x)

## S3 method for class 'aat_splithalf'
plot(x, type = c("median", "minimum", "maximum", "random"))

prune_nothing(ds, ...)

trial_prune_percent_subject(
  ds,
  subjvar,
  rtvar,
  lowerpercent = 0.01,
  upperpercent = 0.99,
  ...
)

trial_prune_percent_sample(
  ds,
  rtvar,
  lowerpercent = 0.01,
  upperpercent = 0.99,
  ...
)

trial_prune_3SD(ds, subjvar, rtvar, ...)

trial_prune_SD_dropcases(
  ds,
  subjvar,
  rtvar,
  trialsd = 3,
  maxoutliers = 0.15,
  ...
)

trial_recode_SD(ds, subjvar, rtvar, trialsd = 3, ...)

case_prune_3SD(ds, ...)

error_replace_blockmeanplus(
  ds,
  subjvar,
  rtvar,
  blockvar,
  errorvar,
  errorbonus,
  ...
)

error_prune_dropcases(ds, subjvar, errorvar, maxerrors = 0.15, ...)
\end{verbatim}
\end{Usage}
%
\begin{Arguments}
\begin{ldescription}
\item[\code{ds}] a longformat data.frame

\item[\code{subjvar}] Quoted name of the participant identifier column

\item[\code{pullvar}] Quoted name of the column indicating pull trials.
Pull trials should either be represented by 1, or by the second level of a factor.

\item[\code{targetvar}] Name of the column indicating trials featuring the target stimulus.
Target stimuli should either be represented by 1, or by the second level of a factor.

\item[\code{rtvar}] Name of the reaction time column.

\item[\code{iters}] Total number of desired iterations. At least 200 are recommended for reasonable confidence intervals;
If you want to see plots of your data, 1 iteration is enough.

\item[\code{plot}] Create a scatterplot of the AAT scores computed from each half of the data from the last iteration.
This is highly recommended, as it helps to identify outliers that can inflate or diminish the reliability.

\item[\code{include.raw}] logical indicating whether raw split-half data should be included in the output object.

\item[\code{cluster}] pre-specified registered multi-core DoParallel cluster that can be used to speed up computations if multiple calls to aat\_splithalf are made.
If no cluster is provided, aat\_splithalf will start up a cluster each time it is called, which takes some extra time.

\item[\code{algorithm}] Function (without brackets or quotes) to be used to compute AAT scores. See \LinkA{Algorithms}{Algorithms} for a list of usable algorithms.

\item[\code{trialdropfunc}] Function (without brackets or quotes) to be used to exclude outlying trials in each half.
The way you handle outliers for the reliability computation should mimic the way you do it in your regular analyses.
It is recommended to exclude outlying trials when computing AAT scores using the mean double-dfference scores and regression scoring approaches,
but not when using d-scores or median double-difference scores.
\begin{itemize}

\item{} \code{prune\_nothing} excludes no trials (default)
\item{} \code{trial\_prune\_3SD} excludes trials deviating more than 3SD from the mean per participant.
\item{} \code{trial\_prune\_SD\_dropcases} removes trials deviating more than a specific number of standard deviations from the participant's mean,
and removes participants with an excessive percentage of outliers.
Required arguments:
\begin{itemize}

\item{} \code{trialsd} - trials deviating more than \code{trialsd} standard deviations from the participant's mean are excluded (optional; default is 3)
\item{} \code{maxoutliers} - participants with a higher percentage of outliers are removed from the data. (optional; default is .15)

\end{itemize}

\item{} \code{trial\_recode\_SD} recodes outlying reaction times to the nearest non-outlying value,
with outliers defined as reaction times deviating more than a certain number of standard deviations from the participant's mean. Required argument:
\begin{itemize}

\item{} \code{trialsd} - trials deviating more than this many standard deviations from the mean are classified as outliers.

\end{itemize}

\item{} \code{trial\_prune\_percent\_subject} and \code{trial\_prune\_percent\_sample} remove trials below and/or above certain percentiles,
on a subject-by-subject basis or sample-wide, respectively. The following arguments are available:
\begin{itemize}

\item{} \code{lowerpercent} and \code{uppperpercent} (optional; defaults are .01 and .99).

\end{itemize}


\end{itemize}


\item[\code{errortrialfunc}] Function (without brackets or quotes) to apply to an error trial.

\begin{itemize}

\item{} \code{prune\_nothing} removes no errors (default).
\item{} \code{error\_replace\_blockmeanplus} replaces error trial reaction times with the block mean, plus an arbitrary extra quantity.
If used, the following additional arguments are required:
\begin{itemize}

\item{} \code{blockvar} - Quoted name of the block variable (mandatory)
\item{} \code{errorvar} - Quoted name of the error variable, where errors are 1 or TRUE and correct trials are 0 or FALSE (mandatory)
\item{} \code{errorbonus} - Amount to add to the reaction time of error trials. Default is 0.6 (recommended by \code{Greenwald, Nosek, \& Banaji, 2003})

\end{itemize}

\item{} \code{error\_prune\_dropcases} removes errors and drops participants if they have more errors than a given percentage. The following arguments are available:
\begin{itemize}

\item{} \code{errorvar} - Quoted name of the error variable, where errors are 1 or TRUE and correct trials are 0 or FALSE (mandatory)
\item{} \code{maxerrors} - participants with a higher percentage of errors are excluded from the dataset. Default is .15.

\end{itemize}


\end{itemize}


\item[\code{casedropfunc}] Function (without brackets or quotes) to be used to exclude outlying participant scores in each half.
The way you handle outliers here should mimic the way you do it in your regular analyses.
\begin{itemize}

\item{} \code{prune\_nothing} excludes no participants (default)
\item{} \code{case\_prune\_3SD} excludes participants deviating more than 3SD from the sample mean.

\end{itemize}


\item[\code{...}] Other arguments, to be passed on to the algorithm or outlier rejection functions (see arguments above)

\item[\code{x}] an \code{aat\_splithalf} object

\item[\code{type}] Character argument indicating which iteration should be chosen. Must be an abbreviation of
\code{"median"} (default), \code{"minimum"}, \code{"maximum"}, or \code{"random"}.
\end{ldescription}
\end{Arguments}
%
\begin{Value}
A list, containing the mean bootstrapped split-half reliability, bootstrapped 95
a list of data.frames used over each iteration, and a vector containing the split-half reliability of each iteration.
\end{Value}
%
\begin{Author}\relax
Sercan Kahveci
\end{Author}
%
\begin{SeeAlso}\relax
\LinkA{q\_reliability}{q.Rul.reliability}
\end{SeeAlso}
%
\begin{Examples}
\begin{ExampleCode}
#Not Run
aat_splithalf(ds=ds2,subjvar="subjectid",pullvar="is_pull",targetvar="is_food",
              rtvar="rt",iters=1000,trialdropfunc=trial_prune_3SD,
              casedropfunc=case_prune_3SD,plot=T,algorithm=aat_dscore)
#Mean reliability: 0.521959
#Spearman-Brown-corrected r: 0.6859041
#95%CI: [0.4167018, 0.6172474]

#Regression Splithalf
aat_splithalf(ds=ds2,subjvar="subjectid",pullvar="is_pull",targetvar="is_food",
              rtvar="rt",iters=100,trialdropfunc=trial_prune_3SD,
              casedropfunc=case_prune_3SD,plot=T,algorithm=aat_regression,
              formula = "rt ~ is_pull * is_food",
              aatterm = "is_pull:is_food")
#Mean reliability: 0.5313939
#Spearman-Brown-corrected r: 0.6940003
#95%CI: [0.2687186, 0.6749176]
#Coming soon
\end{ExampleCode}
\end{Examples}
\inputencoding{utf8}
\HeaderA{Algorithms}{AAT score computation algorithms}{Algorithms}
\aliasA{aat\_doublemeandiff}{Algorithms}{aat.Rul.doublemeandiff}
\aliasA{aat\_doublemeanquotient}{Algorithms}{aat.Rul.doublemeanquotient}
\aliasA{aat\_doublemediandiff}{Algorithms}{aat.Rul.doublemediandiff}
\aliasA{aat\_doublemedianquotient}{Algorithms}{aat.Rul.doublemedianquotient}
\aliasA{aat\_dscore}{Algorithms}{aat.Rul.dscore}
\aliasA{aat\_dscore\_multiblock}{Algorithms}{aat.Rul.dscore.Rul.multiblock}
\aliasA{aat\_regression}{Algorithms}{aat.Rul.regression}
\aliasA{aat\_singlemeandiff}{Algorithms}{aat.Rul.singlemeandiff}
\aliasA{aat\_singlemediandiff}{Algorithms}{aat.Rul.singlemediandiff}
\aliasA{aat\_standardregression}{Algorithms}{aat.Rul.standardregression}
%
\begin{Description}\relax
\begin{itemize}

\item{} \code{aat\_doublemeandiff} computes a mean-based double-difference score:

\code{(mean(push\_target) - mean(pull\_target)) - (mean(push\_control) - mean(pull\_control))}
\item{} \code{aat\_doublemediandiff} computes a median-based double-difference score:

\code{(median(push\_target) - median(pull\_target)) - (median(push\_control) - median(pull\_control))}
\item{} \code{aat\_dscore} computes D-scores for a 2-block design (see Greenwald, Nosek, and Banaji, 2003):

\code{((mean(push\_target) - mean(pull\_target)) - (mean(push\_control) - mean(pull\_control))) / sd(participant\_reaction\_times)}
\item{} \code{aat\_dscore\_multiblock} computes D-scores for pairs of sequential blocks
and averages the resulting score (see Greenwald, Nosek, and Banaji, 2003).
Requires extra \code{blockvar} argument, indicating the name of the block variable.
\item{} \code{aat\_regression} and \code{aat\_standardregression} fit regression models to participants' reaction times and extract a term that serves as AAT score.
\code{aat\_regression} extracts the raw coefficient, equivalent to a mean difference score.
\code{aat\_standardregression} extracts the t-score of the coefficient, standardized on the basis of the variability of the participant's reaction times.
These algorithms can be used to regress nuisance variables out of the data before computing AAT scores.
When using these functions, additional arguments must be provided:
\begin{itemize}

\item{} \code{formula} - a quoted formula to fit to the data;
\item{} \code{aatterm} - the quoted random effect within the subject variable that indicates the approach bias; this is usually the interaction of the pull and target terms.

\end{itemize}

\item{} \code{aat\_doublemeanquotient} and \code{aat\_doublemedianquotient} compute a log-transformed ratio of approach to avoidance for both stimulus categories and subtract these ratios:

\code{log(mean(pull\_target) / mean(push\_target)) - log(mean(pull\_control) / mean(push\_control))}
\item{} \code{aat\_singlemeandiff} and \code{aat\_singlemediandiff} subtract the mean or median approach reaction time from the mean or median avoidance reaction time.
These algorithms are only sensible if the supplied data contain a single stimulus category.

\end{itemize}

\end{Description}
%
\begin{Usage}
\begin{verbatim}
aat_doublemeandiff(ds, subjvar, pullvar, targetvar, rtvar, ...)

aat_doublemediandiff(ds, subjvar, pullvar, targetvar, rtvar, ...)

aat_dscore(ds, subjvar, pullvar, targetvar, rtvar, ...)

aat_dscore_multiblock(ds, subjvar, pullvar, targetvar, rtvar, blockvar, ...)

aat_regression(ds, subjvar, formula, aatterm, ...)

aat_standardregression(ds, subjvar, formula, aatterm, ...)

aat_doublemedianquotient(ds, subjvar, pullvar, targetvar, rtvar, ...)

aat_doublemeanquotient(ds, subjvar, pullvar, targetvar, rtvar, ...)

aat_singlemeandiff(ds, subjvar, pullvar, rtvar, ...)

aat_singlemediandiff(ds, subjvar, pullvar, rtvar, ...)
\end{verbatim}
\end{Usage}
%
\begin{Arguments}
\begin{ldescription}
\item[\code{ds}] A long-format data.frame

\item[\code{subjvar}] Column name of the participant identifier variable

\item[\code{pullvar}] Column name of the movement variable (0: avoid; 1: approach)

\item[\code{targetvar}] Column name of the stimulus category variable (0: control stimulus; 1: target stimulus)

\item[\code{rtvar}] Column name of the reaction time variable

\item[\code{...}] Other arguments passed on by functions (ignored)

\item[\code{blockvar}] name of the variable indicating block number

\item[\code{formula}] A character string containing a regression formula to fit to the data to compute an AAT score

\item[\code{aatterm}] The formula term representing the approach bias. Usually this is the interaction of the movement-direction and stimulus-category terms.
\end{ldescription}
\end{Arguments}
%
\begin{Value}
A data.frame containing participant number and computed AAT score.
\end{Value}
\inputencoding{utf8}
\HeaderA{erotica}{AAT examining approach bias for erotic stimuli}{erotica}
\keyword{datasets}{erotica}
%
\begin{Description}\relax
AAT
\end{Description}
%
\begin{Usage}
\begin{verbatim}
erotica
\end{verbatim}
\end{Usage}
%
\begin{Format}
An object of class \code{"data.frame"}
\end{Format}
%
\begin{Source}\relax
\Rhref{https://osf.io/6h2rj/}{osf.io repository}
\end{Source}
%
\begin{References}\relax
Kahveci, S., Van Bockstaele, B.D., \& Wiers, R.W. (in preparation).
Pulling for Pleasure? Erotic Approach-Bias Associated With Porn Use, Not Problems. DOI:10.17605/OSF.IO/6H2RJ
\end{References}
\inputencoding{utf8}
\HeaderA{q\_reliability}{Compute psychological experiment reliability}{q.Rul.reliability}
\aliasA{plot.qreliability}{q\_reliability}{plot.qreliability}
%
\begin{Description}\relax
This function can be used to compute an exact reliability score for a psychological task whose results involve a difference score.
The resulting q coefficient is equivalent to the average all possible split-half reliability scores.
\end{Description}
%
\begin{Usage}
\begin{verbatim}
q_reliability(ds, subjvar, formula, aatterm = NA)

## S3 method for class 'qreliability'
plot(x)
\end{verbatim}
\end{Usage}
%
\begin{Arguments}
\begin{ldescription}
\item[\code{ds}] a long-format data.frame

\item[\code{subjvar}] name of the subject variable

\item[\code{formula}] a formula predicting the participant's reaction time using trial-level variables such as movement direction and stimulus category

\item[\code{aatterm}] a string denoting the term in the formula that contains the participant's approach bias
\end{ldescription}
\end{Arguments}
%
\begin{Value}
a qreliability object, containing the reliability coefficient, and a data.frame with participants' bias scores and score variance.
\end{Value}
%
\begin{Author}\relax
Sercan Kahveci
\end{Author}
%
\begin{Examples}
\begin{ExampleCode}
q_reliability(ds=dataset,subjvar="subjectid",formula= rt ~ is_pull * is_food, aatterm = "is_pull:is_food")

\end{ExampleCode}
\end{Examples}
\inputencoding{utf8}
\HeaderA{SpearmanBrown}{Spearman-Brown corrections for Correlation Coefficients}{SpearmanBrown}
%
\begin{Description}\relax
Perform a Spearman-Brown correction on the provided correlation score.
\end{Description}
%
\begin{Usage}
\begin{verbatim}
SpearmanBrown(
  corr,
  ntests = 2,
  fix.negative = c("nullify", "bilateral", "none")
)
\end{verbatim}
\end{Usage}
%
\begin{Arguments}
\begin{ldescription}
\item[\code{corr}] To-be-corrected correlation coefficient

\item[\code{ntests}] An integer indicating how many times larger the full test is, for which the corrected correlation coefficient is being computed.
When \code{ntests=2}, the formula will compute what the correlation coefficient would be if the test were twice as long.

\item[\code{fix.negative}] Determines how to deal with a negative value. "nullify" sets it to zero,
"bilateral" applies the correction as if it were a positive number, and then sets it to negative. "none" gives the raw value.
\end{ldescription}
\end{Arguments}
%
\begin{Details}\relax
Correct a correlation coefficient for being based on only a subset of the data.
\end{Details}
%
\begin{Value}
Spearman-Brown-corrected correlation coefficient.
\end{Value}
\printindex{}
\end{document}
